\documentclass[11pt,a4paper]{article} % 문서종류의 설정

\linespread{1.3} % 행간길이
\addtolength{\voffset}{-2cm} % 머릿말여백
\addtolength{\hoffset}{-2cm} % 좌측여백
\addtolength{\textwidth}{4cm} % 본문길이
\addtolength{\textheight}{4cm} % 본문높이

\usepackage{kotex} %kotex 사용
\usepackage{amssymb} % 수식 사용
\usepackage[english]{babel}
\usepackage{amsmath} %수식 시리즈
\usepackage{amsfonts} %수식 시리즈
\usepackage{amssymb} %수식 시리즈
\usepackage{graphicx} %includegraphics
\usepackage{mathrsfs} %수식 중 특수문자
\usepackage{times}
\usepackage[T1]{fontenc}
\usepackage[all]{xy} %xy-pic

\begin{document} % 문서의 시작
\title{제목을입력하시오}
\author{최정훈}
\maketitle
\begin{abstract}
내용을 입력하시오
\end{abstract}
\newpage

\tableofcontents
\newpage

\section{서론}
서론내용+subsections
\newpage

\section{본론}
본론내용 + subsections
\newpage

\section{결론}
결론내용
\newpage

\begin{thebibliography}{99} % Reference 시작
\bibitem{pa} 저자, 제목,저널,연도,(페이지)  %working paper면 기관 Working paper 라고 쓸것, 페이지 수 적으려면 연도에는 괄호를 칠것
% Reference 종료

\end{thebibliography}

\end{document} % 문서 끝
